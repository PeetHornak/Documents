%==============================================================================
% tento soubor pouzijte jako zaklad
% this file should be used as a base for the thesis
% Autoři / Authors: 2008 Michal Bidlo, 2019 Jaroslav Dytrych
% Kontakt pro dotazy a připomínky: sablona@fit.vutbr.cz
% Contact for questions and comments: sablona@fit.vutbr.cz
%==============================================================================
% kodovani: UTF-8 (zmena prikazem iconv, recode nebo cstocs)
% encoding: UTF-8 (you can change it by command iconv, recode or cstocs)
%------------------------------------------------------------------------------
% zpracování / processing: make, make pdf, make clean
%==============================================================================
% Soubory, které je nutné upravit nebo smazat: / Files which have to be edited or deleted:
%   projekt-20-literatura-bibliography.bib - literatura / bibliography
%   projekt-01-kapitoly-chapters.tex - obsah práce / the thesis content
%   projekt-01-kapitoly-chapters-en.tex - obsah práce v angličtině / the thesis content in English
%   projekt-30-prilohy-appendices.tex - přílohy / appendices
%   projekt-30-prilohy-appendices-en.tex - přílohy v angličtině / appendices in English
%==============================================================================
\documentclass[slovak]{fitthesis} % bez zadání - pro začátek práce, aby nebyl problém s překladem
%\documentclass[english]{fitthesis} % without assignment - for the work start to avoid compilation problem
%\documentclass[zadani]{fitthesis} % odevzdani do wisu a/nebo tisk s barevnými odkazy - odkazy jsou barevné
%\documentclass[english,zadani]{fitthesis} % for submission to the IS FIT and/or print with color links - links are color
%\documentclass[zadani,print]{fitthesis} % pro černobílý tisk - odkazy jsou černé
%\documentclass[english,zadani,print]{fitthesis} % for the black and white print - links are black
%\documentclass[zadani,cprint]{fitthesis} % pro barevný tisk - odkazy jsou černé, znak VUT barevný
%\documentclass[english,zadani,cprint]{fitthesis} % for the print - links are black, logo is color
% * Je-li práce psaná v anglickém jazyce, je zapotřebí u třídy použít 
%   parametr english následovně:
%   If thesis is written in English, it is necessary to use 
%   parameter english as follows:
%      \documentclass[english]{fitthesis}
% * Je-li práce psaná ve slovenském jazyce, je zapotřebí u třídy použít 
%   parametr slovak následovně:
%   If the work is written in the Slovak language, it is necessary 
%   to use parameter slovak as follows:
%      \documentclass[slovak]{fitthesis}
% * Je-li práce psaná v anglickém jazyce se slovenským abstraktem apod., 
%   je zapotřebí u třídy použít parametry english a enslovak následovně:
%   If the work is written in English with the Slovak abstract, etc., 
%   it is necessary to use parameters english and enslovak as follows:
%      \documentclass[english,enslovak]{fitthesis}

% Základní balíčky jsou dole v souboru šablony fitthesis.cls
% Basic packages are at the bottom of template file fitthesis.cls
% zde můžeme vložit vlastní balíčky / you can place own packages here

% Kompilace po částech (rychlejší, ale v náhledu nemusí být vše aktuální)
% Compilation piecewise (faster, but not all parts in preview will be up-to-date)
% \usepackage{subfiles}

% Nastavení cesty k obrázkům
% Setting of a path to the pictures
%\graphicspath{{obrazky-figures/}{./obrazky-figures/}}
%\graphicspath{{obrazky-figures/}{../obrazky-figures/}}

%---rm---------------
\renewcommand{\rmdefault}{lmr}%zavede Latin Modern Roman jako rm / set Latin Modern Roman as rm
%---sf---------------
\renewcommand{\sfdefault}{qhv}%zavede TeX Gyre Heros jako sf
%---tt------------
\renewcommand{\ttdefault}{lmtt}% zavede Latin Modern tt jako tt

% vypne funkci šablony, která automaticky nahrazuje uvozovky,
% aby nebyly prováděny nevhodné náhrady v popisech API apod.
% disables function of the template which replaces quotation marks
% to avoid unnecessary replacements in the API descriptions etc.
\csdoublequotesoff



\usepackage{url}
\usepackage{pifont}


% =======================================================================
% balíček "hyperref" vytváří klikací odkazy v pdf, pokud tedy použijeme pdflatex
% problém je, že balíček hyperref musí být uveden jako poslední, takže nemůže
% být v šabloně
% "hyperref" package create clickable links in pdf if you are using pdflatex.
% Problem is that this package have to be introduced as the last one so it 
% can not be placed in the template file.
\ifWis
\ifx\pdfoutput\undefined % nejedeme pod pdflatexem / we are not using pdflatex
\else
  \usepackage{color}
  \usepackage[unicode,colorlinks,hyperindex,plainpages=false,pdftex]{hyperref}
  \definecolor{hrcolor-ref}{RGB}{223,52,30}
  \definecolor{hrcolor-cite}{HTML}{2F8F00}
  \definecolor{hrcolor-urls}{HTML}{092EAB}
  \hypersetup{
	linkcolor=hrcolor-ref,
	citecolor=hrcolor-cite,
	filecolor=magenta,
	urlcolor=hrcolor-urls
  }
  \def\pdfBorderAttrs{/Border [0 0 0] }  % bez okrajů kolem odkazů / without margins around links
  \pdfcompresslevel=9
\fi
\else % pro tisk budou odkazy, na které se dá klikat, černé / for the print clickable links will be black
\ifx\pdfoutput\undefined % nejedeme pod pdflatexem / we are not using pdflatex
\else
  \usepackage{color}
  \usepackage[unicode,colorlinks,hyperindex,plainpages=false,pdftex,urlcolor=black,linkcolor=black,citecolor=black]{hyperref}
  \definecolor{links}{rgb}{0,0,0}
  \definecolor{anchors}{rgb}{0,0,0}
  \def\AnchorColor{anchors}
  \def\LinkColor{links}
  \def\pdfBorderAttrs{/Border [0 0 0] } % bez okrajů kolem odkazů / without margins around links
  \pdfcompresslevel=9
\fi
\fi
% Řešení problému, kdy klikací odkazy na obrázky vedou za obrázek
% This solves the problems with links which leads after the picture
\usepackage[all]{hypcap}

% Informace o práci/projektu / Information about the thesis
%---------------------------------------------------------------------------
\projectinfo{
  %Prace / Thesis
  project={BP},            %typ práce BP/SP/DP/DR  / thesis type (SP = term project)
  year={2020},             % rok odevzdání / year of submission
  date=\today,             % datum odevzdání / submission date
  %Nazev prace / thesis title
  title.cs={Detekce anomálií v síťovém provozu},  % název práce v češtině či slovenštině (dle zadání) / thesis title in czech language (according to assignment)
  title.en={}, % název práce v angličtině / thesis title in english
  %title.length={14.5cm}, % nastavení délky bloku s titulkem pro úpravu zalomení řádku (lze definovat zde nebo níže) / setting the length of a block with a thesis title for adjusting a line break (can be defined here or below)
  %sectitle.length={14.5cm}, % nastavení délky bloku s druhým titulkem pro úpravu zalomení řádku (lze definovat zde nebo níže) / setting the length of a block with a second thesis title for adjusting a line break (can be defined here or below)
  %Autor / Author
  author.name={Peter},   % jméno autora / author name
  author.surname={Horňák},   % příjmení autora / author surname
  %author.title.p={Bc.}, % titul před jménem (nepovinné) / title before the name (optional)
  %author.title.a={Ph.D.}, % titul za jménem (nepovinné) / title after the name (optional)
  %Ustav / Department
  department={UIFS}, % doplňte příslušnou zkratku dle ústavu na zadání: UPSY/UIFS/UITS/UPGM / fill in appropriate abbreviation of the department according to assignment: UPSY/UIFS/UITS/UPGM
  % Školitel / supervisor
  supervisor.name={},   % jméno školitele / supervisor name 
  supervisor.surname={},   % příjmení školitele / supervisor surname
  supervisor.title.p={},   %titul před jménem (nepovinné) / title before the name (optional)
  supervisor.title.a={},    %titul za jménem (nepovinné) / title after the name (optional)
  % Klíčová slova / keywords
  keywords.cs={Bezpečnosť, JavaScript Restrictor, Chrome Zero, JavaScript Zero, rozšírenie webového prehliadača, mikroarchitektúrne útoky, časovače s vysokým rozlíšením, web}, % klíčová slova v českém či slovenském jazyce / keywords in czech or slovak language
  keywords.en={Security, JavaScript Restrictor, Chrome Zero, JavaScript Zero, web browser extension, microarchitectural attacks, high resolution timers, web}, % klíčová slova v anglickém jazyce / keywords in english
  %keywords.en={Here, individual keywords separated by commas will be written in English.},
  % Abstrakt / Abstract
  abstract.cs={Táto práca sa zaoberá zlepšovaním bezpečnosti a ochranou súkromia užívateľov webových prehliadačov rozšírením funkcionality doplnku webového prehliadača nazývaného JavaScript Restrictor, ktorého prototyp bol vytvorený za účelom ochrany užívateľov na webe. V~rámci tejto práce sú analyzované bezpečnostné opatrenia realizované nástrojom Chrome Zero, ktorý implementuje opatrenia proti mikroarchitektúrnym útokom, útokom zneužívajúcich časovače s vysokým rozlíšením a zneužitiu samotného JavaScript enginu. Práca vyhodnocuje aktuálnosť opatrení a vybrané z nich integruje do JavaScript Restrictoru. Opatrenia sú otestované a vyhodnotené z pohľadu každodenného používania.}, % abstrakt v českém či slovenském jazyce / abstract in czech or slovak language
  abstract.en={This thesis deals with improvements of security and privacy protection of web browser users by enhancing functionality of web browser extension called JavaScript Restrictor, which prototype was created in order to protect users on web. Within this thesis are analyzed security measures realized by tool Chrome Zero, which implements measures against microarchitectural attacks, attacks abusing high resolution timers and abuse of JavaScript engine. Thesis evaulates topicality of measures and integrates selected into JavaScript Restrictor. Measures are tested and evaluated from every day use point of view.}, % abstrakt v anglickém jazyce / abstract in english
  %abstract.en={An abstract of the work in English will be written in this paragraph.},
  % Prohlášení (u anglicky psané práce anglicky, u slovensky psané práce slovensky) / Declaration (for thesis in english should be in english)
  declaration={Prehlasujem, že som túto bakalársku prácu vypracoval samostatne pod vedením pána Ing. Libora Polčáka Ph.D.
Uviedol som všetky literárne pramene, publikácie a ďalšie zdroje, z~ktorých som čerpal.},
  %declaration={I hereby declare that this Bachelor's thesis was prepared as an original work by the author under the supervision of Mr. X
% The supplementary information was provided by Mr. Y
% I have listed all the literary sources, publications and other sources, which were used during the preparation of this thesis.},
  % Poděkování (nepovinné, nejlépe v jazyce práce) / Acknowledgement (optional, ideally in the language of the thesis)
  acknowledgment={Rád by som poďakoval Ing. Liborovi Polčákovi PhD. za jeho čas, ktorý venoval pri vedení tejto bakalárskej práce a za poskytnutú odbornú pomoc. Tak isto by som rád poďakoval mojim spolužiakom a kamarátom, ktorí ma motivovali pri písaní tejto práce. Nakoniec by som chcel poďakovať mojim najbližším, ktorí ma vždy podporili, keď som to potreboval.},
  %acknowledgment={Here it is possible to express thanks to the supervisor and to the people which provided professional help
%(external submitter, consultant, etc.).},
  % Rozšířený abstrakt (cca 3 normostrany) - lze definovat zde nebo níže / Extended abstract (approximately 3 standard pages) - can be defined here or below
  %extendedabstract={Do tohoto odstavce bude zapsán rozšířený výtah (abstrakt) práce v českém (slovenském) jazyce.},
  %faculty={FIT}, % FIT/FEKT/FSI/FA/FCH/FP/FAST/FAVU/USI/DEF
  faculty.cs={Fakulta informačních technologií}, % Fakulta v češtině - pro využití této položky výše zvolte fakultu DEF / Faculty in Czech - for use of this entry select DEF above
  faculty.en={Faculty of Information Technology}, % Fakulta v angličtině - pro využití této položky výše zvolte fakultu DEF / Faculty in English - for use of this entry select DEF above
  department.cs={Ústav matematiky}, % Ústav v češtině - pro využití této položky výše zvolte ústav DEF nebo jej zakomentujte / Department in Czech - for use of this entry select DEF above or comment it out
  department.en={Institute of Mathematics} % Ústav v angličtině - pro využití této položky výše zvolte ústav DEF nebo jej zakomentujte / Department in English - for use of this entry select DEF above or comment it out
}

% Rozšířený abstrakt (cca 3 normostrany) - lze definovat zde nebo výše / Extended abstract (approximately 3 standard pages) - can be defined here or above
%\extendedabstract{Do tohoto odstavce bude zapsán výtah (abstrakt) práce v českém (slovenském) jazyce.}

% nastavení délky bloku s titulkem pro úpravu zalomení řádku - lze definovat zde nebo výše / setting the length of a block with a thesis title for adjusting a line break - can be defined here or above
%\titlelength{14.5cm}
% nastavení délky bloku s druhým titulkem pro úpravu zalomení řádku - lze definovat zde nebo výše / setting the length of a block with a second thesis title for adjusting a line break - can be defined here or above
%\sectitlelength{14.5cm}

% řeší první/poslední řádek odstavce na předchozí/následující stránce
% solves first/last row of the paragraph on the previous/next page
\clubpenalty=10000
\widowpenalty=10000

% checklist
\newlist{checklist}{itemize}{1}
\setlist[checklist]{label=$\square$}

\begin{document}
  % Vysazeni titulnich stran / Typesetting of the title pages
  % ----------------------------------------------
  \maketitle
  % Obsah
  % ----------------------------------------------
  \setlength{\parskip}{0pt}

  {\hypersetup{hidelinks}\tableofcontents}
  
  % Seznam obrazku a tabulek (pokud prace obsahuje velke mnozstvi obrazku, tak se to hodi)
  % List of figures and list of tables (if the thesis contains a lot of pictures, it is good)
  \ifczech
    \renewcommand\listfigurename{Seznam obrázků}
  \fi
  \ifslovak
    \renewcommand\listfigurename{Zoznam obrázkov}
  \fi
  % {\hypersetup{hidelinks}\listoffigures}
  
  \ifczech
    \renewcommand\listtablename{Seznam tabulek}
  \fi
  \ifslovak
    \renewcommand\listtablename{Zoznam tabuliek}
  \fi
  % {\hypersetup{hidelinks}\listoftables}

  \ifODSAZ
    \setlength{\parskip}{0.5\bigskipamount}
  \else
    \setlength{\parskip}{0pt}
  \fi

  % vynechani stranky v oboustrannem rezimu
  % Skip the page in the two-sided mode
  \iftwoside
    \cleardoublepage
  \fi

  % Text prace / Thesis text
  % ----------------------------------------------
  \ifenglish
    \input{projekt-01-kapitoly-chapters-en}
  \else
    \chapter{Úvod}
\label{Sec1}

S rastúcou popularitou a pokrokom sieťových technológií, internetové služby, ktoré sú poskytované komerčnými, neziskovými a štátnymi organizáciami podstupujú konštantný rast a tým pádom spôsobujú zväčšovanie sieťového prenosu \cite{anomaly-book}.

Spoločne s týmto rastom je v dnešnej dobe možné vidieť zneužívanie internetu na rôzne účely. Anomálie ako červy, skenovanie portov, útoky typu denial of service a rôzne iné, je možné vidieť bežne na sieťovej prevádzke. Tieto anomálie mrhajú sieťovými zdrojmi, čo spôsobuje degradáciu výkonu sieťových zariadení a koncových užívateľov a vedú k bezpečnostným problémom \cite{Gu-McCallum-Detecting}.

Preto je dôležité zaoberať sa efektívnym spôsobom monitorovania siete a detekcie útokov, ktoré sa v nej vyskytujú. Čím bližšie sa podarí detegovanie útokov posunúť k detekcii v reálnom čase, tým efektívnejšie a účinnejšie je možné na útoky reagovať. Z toho dôvodu sieťový monitoring neoddeliteľnou súčasťou správy počítačovej siete. Nemenej dôležité je taktiež zníženie doby, uplynutej od výskytu útoku, po jeho úspešné odhalenie \cite{Pavuk:2017:MuniBP}.

Monitorovanie siete je funkcia zberu informácií správy siete. Aplikácie na monitorovanie siete sa vytvárajú na zhromažďovanie údajov pre aplikácie na správu siete. Hlavným cieľom sieťového monitoringu je zbieranie použiteľných dát z viacerých častí siete, takým spôsobom, že vďaka pozberaným dátam môže byť sieť kontrolovaná a spravovaná \cite{PDF-Monitoring}. Keďže stále viac a viac sieťových zariadení sa pripája do siete, čím vznikajú ešte väčšie siete, spoločne s tým techniky pre monitorovanie sa stávajú nutnou súčasťou sietí.

Táto práca popisuje viaceré techniky a prístupy, ktoré je možné využiť pre detekciu anomálií v sieťovom prenose.

\chapter{Detekčný systém vniknutia}

Detekčný systém vniknutia z anglického výrazu intrusion detection system (NIDS), je systém využívaný pre monitorovanie siete a určený na detegovanie možných prienikov do siete spôsobujúcich škodlivú činnosť, počítačové útoky alebo zneužitie počítačov vírusom a následným upozornením správcov po detekcií \cite{NIDS-Kumar}. Systém NIDS monitoruje a analyzuje dátové pakety, ktoré vstupujú do siete, hladaj
úc podozrivé aktivity. Väčšie NIDS systémy môžu byť nasadené na uzloch chrbtovej sieti, pre monitorovanie veľkého množstva prevádzky. Menšie z nich je možné nasadiť na špecifický server, smerovač, bránu alebo router. 
\newline
\newline
NIDS systémy sú klasifikované do 3 hlavných kategórií:
\begin{itemize}
    \item Detekčný mechanizmus založený na príznakoch
    \item Detekčný mechanizmus založený na analýze stavových protokolov
    \item Detekčný mechanizmus založený na anomáliách
\end{itemize}

\subsection*{Detekčný mechanizmus založený na príznakoch}
Príznak je vzorka, ktorá odpovedá známej hrozbe. Detekcia založená na príznakoch je proces porovnávania príznakov oproti pozorovaným udalostiam s cieľom identifikovať možné incidenty \cite{NIDS-Hung-Chun}. Keďže sa používajú znalosti získané na základe špecifických útokov a slabostí systému, tento spôsob detekcie sa nazýva aj Detekčný mechanizmus založený na znalostiach. 

Tento mechanizmus je veľmi účinný pri detekcii známych hrozieb, ale neefektívny pri detekcii doteraz neznámych hrozieb, poprípade pri detekcií starých hrozieb využívajúce mechanizmus vyhýbania. Je to najjednoduchšia metóda, pretože iba porovnáva súčasné jednotky aktivity, buď pakety alebo položky v záznamoch so zoznamom príznakov použitím operácie porovnania reťazcov \cite{NIDS-PDF}.

\subsection*{Detekčný mechanizmus založený na analýze stavových protokolov}
Tento mechanizmus je proces porovnania dopredu určených profilov všeobecne akceptovaných definícií povolenej aktivity protokolu pre každý stav protokolu oproti sledovaným udalostiam s cieľom identifikovať odchýlky a tým pádom potenciálne škodlivé stavy \cite{NIDS-Hung-Chun}. Vo väčšine prípadov sú definície profilov sieťových protokolov založené na štandardizačných dokumentoch vytvorených Medzinárodnými štandardizačnými agentúrami. 

Je schopný identifikovať neočakávané postupnosti príkazov ako je opakované zadanie toho istého príkazu alebo zadanie príkazu bez predchádzajúceho zadanie príkazu, na ktorom je závislý. Primárnou nevýhodou tohto mechanizmu je jeho náročnosť na výpočtové zdroje, pretože pre každý protokol musí vytvoriť novú inštanciu stavového automatu a teda pri viacerých súčasne monitorovaných spojeniach musí pre každé spojenie vytvoriť novú inštanciu stavového automatu \cite{NIDS-PDF}.

\subsection*{Detekčný mechanizmus založený na anomáliách}

Detegovanie na základe hľadania anomálií je založené na preddefinovaní klasického správania siete. V prípade, že aktuálne správanie siete nie je v súlade s preddefinovaným správaním, tak mechanizmus spustí udalosť, ktorá sa má vykonať v prípade anomálie \cite{AnomalyDetection}. Špecifikácia akceptovateľného správania je definovaná správcami siete. Tieto špecifikácie sú definované pre správanie rôznych objektov v sieti ako napríklad používatelia, uzly alebo spojenia \cite{NIDS-PDF}.

Hlavnou nevýhodou tohto mechanizmu je potrebné kvalitné špecifikovanie správania a testovanie špecifikácie, na čom záleží priamo závisí efektivita odhaľovania anomálií. Pre správnu detekciu je potrebná perfektná znalosť siete, v opačnom prípade je možné, že anomálie nebudú odhalené alebo budú vyhodnotené ako škodlivé aj v prípade keď nie sú.

Detekcia anomálií má výhodu oproti ostatným mechanizmom v tom, že dokáže odhaliť dovtedy neznáme útoky, pre ktoré ešte nie sú definované ich príznaky. Toto je možné v prípade, že útok sa správa iným spôsobom ako je bežný vzorec prevádzky. Toto je napríklad možné vidieť v prípade, ak je systém nakazený novým druhom červa, ktorý sa ihneď snaží nájsť ďalšie zariadenia, ktoré by mohli byť zraniteľné, čo ihneď zaplní sieť škodlivou komunikáciou a tým pádom systém objaví výchylku v predpokladanom objeme komunikácie. 

\chapter{Techniky pre detekciu anomálií}
\label{Sec3}

S postupom času sa začalo využívať množstvo techník pre objavenie anomálií v sieti. Rozlišujú sa na základe spôsobu spracovania dát, niektoré využívajú klasické štatistické metódy, kognitívne metódy alebo sú založené na umelej inteligencií \cite{Anomaly-Clarke-Yair}. V tejto kapitole sú niektoré z nich popísané.

\section{Štatistické metódy}
Tieto metódy sú vo väčšine prípadov závislé na metrikách dát ako napríklad objem dát v sieti, počet paketov a počet pripojení, ktoré vytvárajú pre každý protokol. Pri týchto modeloch sa správanie vyhodnocuje na základe časových intervalov, v ktorých sa ráta počet udalostí za danú časovú jednotku a následne sa vyhodnocuje poradie a hodnota každej aktivity a ich poradie \cite{AnomalyDetection}.

\subsection*{Markovské procesy}
Medzi Markovské procesy patria takzvané Markovské reťaze, čo je množina stavov, ktoré sú prepojené prechodmi ohodnotenými s určitou pravdepodobnosťou prechodu, ktoré predstavujú topológiu siete a schopnosti modelu. Počas prvej fázy sú pravdepodobnosti prechodov odvodené z klasického správania cieleného systému a následne sú anomálie vyhodnocované pomocou sledovania jednotlivých sekvencií prechodov a porovnávania pravdepodobností s predurčeným prahom \cite{AnomalyMarkovs}.

\subsection*{Štatistické okamihy}
Štatistický priemer, smerodajná odchýlka alebo ostatné štatistické metódy sú označované ako okamihy. Ak vyhodnotenie udalosti je mimo stanovené intervaly, tak sa označí ako anomália. Systém berie do úvahy aj vek dát a na základe toho je nútený upravovať dátové intervaly \cite{anomaly-book}. Pre definovaní tohto modelu nie je potrebné dopredu určiť normálne správanie systému, čo je jedna z hlavných výhod.

\subsection*{Viac rozmerné modely}
Hlavným rozdielom medzi štatistickými okamihmi je, že v tomto prípade sa počíta korelácia medzi dvomi a viacerými metrikami. Tieto modely sa využívajú ak experimentálne dáta ukážu, že dosahujú lepšie výsledky ako v prípade, že sa metriky vyhodnocujú samostatne \cite{anomaly-book}. 

\section{Kognitívne metódy}
Techniky detekcie anomálie založené na poznaní využívajú vstupy od expertov na manuálne zostavenie požadovaného modelu, tento prístup využíva ľudské vstupy na určenie legitímneho správania \cite{Anomaly-Clarke-Yair}.

\subsection*{Konečné automaty}
Tento model využíva konečné automaty, pre analýzu stavov, prechodov a akcií. Stav obsahuje informácie o minulosti. Akcie sú popisy aktivity, ktorá sa má vykonať v danom momente ako napríklad vstup do stavu alebo výstup \cite{AnomalyDetection}. Toto správanie sa následne porovnáva a vyhodnocuje so zostaveným správaním.

\subsection*{Popis skripty}
Sú to skriptovacie jazyky vyvíjané najmä komunitou, ktoré popisujú príznaky útokov a môžu byť použité na detegovanie útokov na základe sekvencie špecifických udalostí \cite{AnomalyDetection}.

\subsection*{Expertné systémy}
V expertných systémoch sa používajú ľudské skúsenosti pri riešení problémov. Systémy riešia nejasnosti, pričom sa obvykle konzultuje s jedným alebo viacerými ľudskými odborníkmi. Tieto systémy sú účinné pri určitých problémoch a tiež sa považujú za skupinu problémov umelej inteligencie \cite{Anomaly-Clarke-Yair}. Expertné systémy sú trénované na základe rozsiahlych znalostí o modeloch spojených so známymi útokmi poskytovanými ľudskými odborníkmi.

\section{Metódy založené na umelej inteligencií}
Techniky strojového učenia sú založené na vytvorení explicitného alebo implicitného modelu, ktorý umožňuje kategorizáciu analyzovaných vzorcov. Čo majú všetky tieto metódy spoločné je potreba ohodnotených údajov na trénovanie modelu správania, čo je postup, ktorý kladie vysoké nároky na zdroje \cite{AnomalyMarkovs}. V mnohých prípadoch sa uplatniteľnosť princípov strojového učenia zhoduje s tou, ktorá platí pre štatistické techniky, hoci prvá sa zameriava na vytvorenie modelu, ktorý zlepšuje jeho výkon na základe predchádzajúcich výsledkov.

\subsection*{Bayesovské siete}
Bayesovská sieť je grafický model, ktorý priraďuje pravdepodobnostný vzťah medzi rôznymi sledovanými premennými. Takýto model môže určiť vzájomné závislosti medzi premennými v prípade malej straty údajov. Okrem toho je schopný tiež predpovedať budúce vzájomné závislosti \cite{NIDS-Kumar}. Hoci sa využitie Bayesovských sietí v určitých situáciách ukázalo ako efektívne, získané výsledky sú vysoko závislé od predpokladov o správaní cieľového systému, a preto odchýlka v týchto hypotézach vedie k chybám pri zisťovaní, ktoré možno pripísať posudzovanému modelu \cite{AnomalyMarkovs}.

\subsection*{Neurónové siete}
Na základe postupnosti príkazov zadaných konkrétnym užívateľom sa systém využívajúci prístup neurónovej siete naučí predpovedať ďalší príkaz, takže neurónové siete riešia problém modelovania správania používateľa v nepretržitom procese, ktorý sa používa pri detekcii anomálie, pretože nie je potrebný žiadny model explicitného použitia \cite{AnomalyDetection}.

\subsection*{Fuzzy logika}
Fuzzy logika je odvodená z teórie fuzzy množín, podľa ktorej je uvažovanie skôr približné ako presne odvodené z klasickej predikátovej logiky. Fuzzy techniky sa preto používajú v oblasti detekcie anomálií hlavne preto, že vlastnosti, ktoré sa majú zvážiť, sa dajú považovať za fuzzy premenné \cite{AnomalyMarkovs}. Hoci sa fuzzy logika ukázala ako účinná, najmä pri skenovaní portov a sondách, jej hlavnou nevýhodou je vysoká spotreba zdrojov.


\chapter{Záver}
Táto práca sa venovala detekčným systémom vniknutia, spôsobom ako sa využívajú, kde sa nasadzujú a ako ich klasifikujeme. Venoval som sa najmä detekčným mechanizmom založených na anomáliách v sieti. V kapitole \ref{Sec3} popisujem 3 hlavné druhy metód, ktoré sa využívajú a princípy, modely a procesy, ktoré jednotlivé metódy implementujú. 

Táto práca mi rozšírila obzory v oblasti bezpečnosti počítačových sietí. Umožnila mi naštudovať si základné princípy detekcie anomálií v sieti a objaviť využitie základných konceptov v informačných technológiách pre analyzovanie komplikovaných a početných dát.
  \fi
  
  % Kompilace po částech (viz výše, nutno odkomentovat)
  % Compilation piecewise (see above, it is necessary to uncomment it)
  %\subfile{projekt-01-uvod-introduction}
  % ...
  %\subfile{chapters/projekt-05-conclusion}


  % Pouzita literatura / Bibliography
  % ----------------------------------------------
\ifslovak
  \makeatletter
  \def\@openbib@code{\addcontentsline{toc}{chapter}{Literatúra}}
  \makeatother
  \bibliographystyle{skplain}
\else
  \ifczech
    \makeatletter
    \def\@openbib@code{\addcontentsline{toc}{chapter}{Literatura}}
    \makeatother
    \bibliographystyle{czplain}
  \else 
    \makeatletter
    \def\@openbib@code{\addcontentsline{toc}{chapter}{Bibliography}}
    \makeatother
    \bibliographystyle{enplain}
  %  \bibliographystyle{alpha}
  \fi
\fi
  \begin{flushleft}
  \bibliography{projekt-20-literatura-bibliography}
  \end{flushleft}

  % vynechani stranky v oboustrannem rezimu
  % Skip the page in the two-sided mode
  \iftwoside
    \cleardoublepage
  \fi

  % Prilohy / Appendices
  % ---------------------------------------------
  \appendix
\ifczech
  \renewcommand{\appendixpagename}{Přílohy}
  \renewcommand{\appendixtocname}{Přílohy}
  \renewcommand{\appendixname}{Příloha}
\fi
\ifslovak
  \renewcommand{\appendixpagename}{Prílohy}
  \renewcommand{\appendixtocname}{Prílohy}
  \renewcommand{\appendixname}{Príloha}
\fi
%  \appendixpage

% vynechani stranky v oboustrannem rezimu
% Skip the page in the two-sided mode
%\iftwoside
%  \cleardoublepage
%\fi
  
\ifslovak
%  \section*{Zoznam príloh}
%  \addcontentsline{toc}{section}{Zoznam príloh}
\else
  \ifczech
%    \section*{Seznam příloh}
%    \addcontentsline{toc}{section}{Seznam příloh}
  \else
%    \section*{List of Appendices}
%    \addcontentsline{toc}{section}{List of Appendices}
  \fi
\fi
  \startcontents[chapters]
  \setlength{\parskip}{0pt} 
  % seznam příloh / list of appendices
  % \printcontents[chapters]{l}{0}{\setcounter{tocdepth}{2}}
  
  \ifODSAZ
    \setlength{\parskip}{0.5\bigskipamount}
  \else
    \setlength{\parskip}{0pt}
  \fi
  
  % vynechani stranky v oboustrannem rezimu
  \iftwoside
    \cleardoublepage
  \fi
  
  
  % Kompilace po částech (viz výše, nutno odkomentovat)
  % Compilation piecewise (see above, it is necessary to uncomment it)
  %\subfile{projekt-30-prilohy-appendices}
  
\end{document}
